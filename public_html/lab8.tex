\documentclass[titlepage][fleqn]{article}
\usepackage[version=3]{mhchem}
\begin{document}
\title{AP Chemistry Lab 8: Analysis of a Mixture}
\author{Bay Foley-Cox \\\\ Lab partners: \\ \\Justin Schaaf, Evan Beal}
\date{Performed on October 5th, 2017}
\maketitle
\section{Introduction}
This lab makes use of the idea that information about an unknown substance can be determined by reacting that substance and measuring its products. In this lab, the unknown substance is a mixture of metal carbonate and metal bicarbonate. The goal is to determine the percent by mass of the bicarbonate. When heated to above 110℃ metal bicarbonates decompose by the following reaction: \ce{2 MHCO3(s) -> M2CO3(s) + H2O(g) + CO2(g)} . Metal carbonates on the other hand will remain stable at temperatures less than 800℃. Since in the equation both water and carbon dioxide are gases, they will escape during heating. Therefore, their mass is determinable by the difference in pre and post heating masses.  This mass can be converted to moles CO2 and H2O. Then, using the stoichiometric ratio from the equation of the decomposition reaction, the moles of bicarbonate can be found as well. This can be converted to the mass of the original bicarbonate. Its percent mass in the mixture can then be found by dividing its mass by the total mass of the initial sample. 

\section{Procedure}
First, the mass of a crucible and lid was found. Then, 1 to 2 grams of the unknown mixture of a metal bicarbonate and a metal carbonate were added to the crucible. The crucible and cover were subsequently reweighed. Next, the crucible was heated on a bunsen burner at a temperature above 110℃ and below 800℃ for five minutes with its lid ajar. Afterwards, it was allowed to cool fully and then was weighed again. This process of heating, allowing to cool and reweighing, was repeated until the change in mass after each heating became negligible. At this point the final mass was recorded.

\section{Data and Observations}

\subsection{Data Collected}
\begin{table}[h]
\def\arraystretch{1.5}
\begin{tabular}{|c|c|}
\hline
 Mass of Crucible and Lid & 23.359g \\ 
\hline
Mass of Crucible, Lid and Mixture & 25.397g\\  
\hline
Mass of crucible, lid, and mixture after first heating & 24.795g\\
\hline
Mass of crucible, lid, and mixture after second heating & 24.793g\\
\hline
\end{tabular}
\end{table}

\subsection{Observations}
\begin{itemize}
  \item The mass of the crucible and its contents only changed by 0.001g, which is withing the balances maragin of error, between the first and second weighing. This indicates that all the sodium bicarbonate was succesfully reacted. 
  \item The contents of the crucible looked identical before and after the heating meaning it was impossible to assess visually whether the reaction had occured. 
\end{itemize}

\section{Calculations}



$$\mbox{Mass mixture} = \mbox{mass crucible, lid, contents} - \mbox{mass crucible, lid}$$
$$Mass mixture = 25.397g  - 23.359g$$
$$Mass mixture = 2.038g$$


$$\Delta mass mixture = mass mixture - (mass crucible, contents, lid after final heating - mass crucible, lid)$$



\section{Conclusion}

The percentage by mass of sodium carbonate in the mixture of sodium bicarbonate and sodium carbonate was 80.28\%. 


\end{document}